\documentclass[12pt]{article}
\usepackage{geometry}                % See geometry.pdf to learn the layout options. There are lots.
\geometry{letterpaper}                   % ... or a4paper or a5paper or ... 
%\geometry{landscape}                % Activate for for rotated page geometry
\usepackage[parfill]{parskip}    % Activate to begin paragraphs with an empty line rather than an indent
\usepackage{daves,fancyhdr,natbib,graphicx,dcolumn,amsmath,lastpage,url}
\usepackage{amsmath,amssymb,epstopdf,longtable}
\DeclareGraphicsRule{.tif}{png}{.png}{`convert #1 `dirname #1`/`basename #1 .tif`.png}
\pagestyle{fancy}
\lhead{CE 3354 -- Engineering Hydrology}
\rhead{FALL 2025}
\lfoot{EX1}
\cfoot{}
\rfoot{Page \thepage\ of \pageref{LastPage}}
\renewcommand\headrulewidth{0pt}



\begin{document}
\begin{center}
{\textbf{{ CE 3354 Engineering Hydrology} \\ {Exam 1}}}
\end{center}

 \section*{\small{Exercises}}
 \begin{enumerate}
%\item Problem 7.3.5 in Mays, pg. 278.

\item problem

\item problem

\item The file ``EVAP2.TXT'' are monthly evaporation rates for somewhere in Texas.  The three locations studied in ES-3, are San Angelo, Dallas, and Houston.

Determine:
\begin{enumerate}
\item A method to decide which location the file most likely to represents.
\item The location (of the three) that the file represents.
\item A plot the monthly rates as a time series (the second column is the series counter).  What do you see in this plot?   Is there any trend (assuming you can remove the periodic component of the signal)?
\item Break the data into three roughly even parts (1950-1969), (1970-1989),(1990-2009).  Perform a non-parametric hypothesis test for equal medians.  Does this data provide any evidence of a change in evaporation depth over time (that is are the median differences significant?)
\item What would be the importance of an increase or decrease in evaporation over time (in the context of climate variability)?
\end{enumerate}

\end{enumerate}

\end{document}  