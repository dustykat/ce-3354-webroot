\documentclass[12pt]{article}
\usepackage{geometry}                % See geometry.pdf to learn the layout options. There are lots.
\geometry{letterpaper}                   % ... or a4paper or a5paper or ... 
%\geometry{landscape}                % Activate for for rotated page geometry
\usepackage[parfill]{parskip}    % Activate to begin paragraphs with an empty line rather than an indent
\usepackage{daves,fancyhdr,natbib,graphicx,dcolumn,amsmath,lastpage,url}
\usepackage{amsmath,amssymb,epstopdf,longtable}
\DeclareGraphicsRule{.tif}{png}{.png}{`convert #1 `dirname #1`/`basename #1 .tif`.png}
\pagestyle{fancy}
\lhead{CE 3354 -- Engineering Hydrology}
\rhead{FALL 2024}
\lfoot{ES5}
\cfoot{}
\rfoot{Page \thepage\ of \pageref{LastPage}}
\renewcommand\headrulewidth{0pt}



\begin{document}
\begin{center}
{\textbf{{ CE 3354 Engineering Hydrology} \\ {Exercise Set 5}}}
\end{center}

 \section*{\small{Exercises}}
 \begin{enumerate}
%\item Problem 7.3.5 in Mays, pg. 278.

\item Estimate the monthly evapotranspiration depths for the San Angelo (Concho County) area using the Blaney-Criddle method.\footnote{A Google search should get you sufficient guidance to perform this exercise.}

\item Estimate the monthly evapotranspiration depths for the San Angelo (Concho County) area using the Thornwaithe method.\footnote{A Google search should get you sufficient guidance to perform this exercise.}

\item How important are these estimates in the drainage analysis project for a storm lasting 24-48 hours?

%\item Estimate the monthly evapotranspiration depths for the Houston area using the Preistly-Taylor method\footnote{You will need to estimate the monthly evapotranspiration depths for Concho County using the Preistly-Taylor method for the semester design project.}

%\item  The file ``EVAP2.TXT'' are monthly evaporation rates for coastal Texas (from Texas Water Development Board).  Assume these are actual evaporation rates for the indicated months.  In the context of questions 2 and 3 perform the following:
%\begin{enumerate}
%\item Compare the long-term monthly averages to the rates you computed in (1) and (2).  Are they the same or different?  Any ideas why?
%\item Plot the monthly rates as a time series (the second column is the series counter).  What do you see in this plot?   Is there any trend (assuming you can remove the periodic component of the signal)?
%\item Break the data into three roughly even parts (1950-1969), (1970-1989),(1990-2009).  Perform a non-parametric hypothesis test for equal medians.  Does this data provide any evidence of a change in evaporation depth over time (that is are the median differences significant?)
%\item What would be the importance of an increase or decrease in evaporation over time (in the context of climate variability)?
%\end{enumerate}
\end{enumerate}

\end{document}  