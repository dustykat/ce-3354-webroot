\documentclass[12pt]{article}

% ------------------------ PAGE LAYOUT ------------------------
\usepackage[letterpaper]{geometry}
\usepackage[parfill]{parskip}    % No paragraph indent, blank line between
\usepackage{fancyhdr}            % Header/Footer formatting
\usepackage{lastpage}            % Reference last page in footer

% ------------------------ MATH & SYMBOLS ---------------------
\usepackage{amsmath, amssymb}

% ------------------------ FIGURES & TABLES -------------------
\usepackage{graphicx}
\usepackage{longtable}          % For multipage tables, if needed
\DeclareGraphicsRule{.tif}{png}{.png}{`convert #1 `dirname #1`/`basename #1 .tif`.png}

% ------------------------ TEXT UTILITIES ---------------------
\usepackage{url}                % For URLs in footnotes
\usepackage{paralist}           % Compact list environments
\usepackage{natbib}             % Bibliographic references
\usepackage[final]{pdfpages}    % For including PDFs

% ------------------------ CUSTOM -------------------------------
\usepackage{daves}

% ------------------------ HEADER / FOOTER ---------------------
\pagestyle{fancy}
\lhead{CE 3354 -- Engineering Hydrology}
\rhead{SUMMER 2025}
\lfoot{ELI5 Assignment}
\cfoot{}
\rfoot{Page \thepage\ of \pageref{LastPage}}
\renewcommand\headrulewidth{0pt}

% ------------------------ DOCUMENT START ---------------------
\begin{document}

\begin{center}
    \textbf{CE 3354 Engineering Hydrology\\
    ELI5 Topic 08}
\end{center}

\section*{Explain Like I’m 5 (ELI5)}

Produce a short written explanation that simplifies a complex hydrologic topic for a non-specialist audience. Your response should reflect the Feynman Method\footnote{\url{http://54.243.252.9/ce-3105-webroot/ce3105notes/_build/html/lessons/laboratory0/feynman/feynman.html#feynman-learning-technique}}, emphasizing clear and concise teaching through explanation.

\textbf{Topic:} What infiltration means (and why it’s like a sponge)

\textbf{Deliverable:}
\begin{itemize}
    \item A one-page typeset explanation (exclusive of images, charts, and references), aimed at a general audience.
\end{itemize}

\end{document}
